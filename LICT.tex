\documentclass[11pt,a4paper]{article}
\usepackage[utf8]{inputenc}
\usepackage[T1]{fontenc}
\usepackage[english]{babel}
\usepackage{amsmath}
\usepackage{amsfonts}
\usepackage{amssymb}
\usepackage{anyfontsize}
\usepackage[final]{graphicx}
\usepackage{makecell}
\usepackage{graphicx}
\usepackage{float}
\usepackage{url}
\usepackage{adjustbox}
\usepackage{svg}
\usepackage{subcaption}
\usepackage[left=2.00cm, right=2.00cm, top=2.00cm, bottom=2.00cm]{geometry}
\author{Niklas Arnold, Dominik Schiwietz, Simon Hofinger}
\title{Laser-Induced Current Transient Technique}
\begin{document}
	\maketitle
	\newpage
	\tableofcontents
	\newpage
	\section{Abstract}
	\section{Scientific Background}
	\section{Experiment}
		\subsection{Experimental set-up}
		\subsection{$ \text{HClO}_4 $ Solution}
		\subsection{Electrode}
		\subsection{LICT measurements}
	\section{Measurement analysis}
	In order to analyze the retrieved data, the peaks of the measured currents are being shown in figure \ref{PME} for three different pH values. In order to find the PME of the Electrode in each specific setup, a fit is applied to the data points. Additionally, the 
	\begin{figure}[H]
		\subfloat[]{\includegraphics[]{Au.01Mol.png}} 
		\subfloat[]{\includegraphics[]{Au.1Mol.png}}\\\\\\
		\subfloat[]{\includegraphics[]{Au1Mol.png}}
		\subfloat[]{\includegraphics[width = .46\textwidth]{jvsEPHAu.png}} 
		\caption{Measured peak currents shown against the applied Voltage for a polycrystalline gold Electrode (a,b,c). Figure (d) shows the pH sensitivity of the current curve against the Voltage. The zero-point of the Voltage is set to the RHE.}
		\label{PME}
	\end{figure}
	\section{Conclusion}
\end{document}